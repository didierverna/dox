% \iffalse                                                 -*- mode: LaTeX -*-
%
% dox.dtx --- Doc file for the DoX package (code and documentation)
%
% Copyright (C) 2009 Didier Verna.
%
% Author:        Didier Verna <didier@lrde.epita.fr>
% Maintainer:    Didier Verna <didier@lrde.epita.fr>
% Created:       Thu Sep 10 21:11:32 2009
% Last Revision: Fri Sep 11 11:20:13 2009
%
% This file is part of DoX.
%
% DoX may be distributed and/or modified under the
% conditions of the LaTeX Project Public License, either version 1.1
% of this license or (at your option) any later version.
% The latest version of this license is in
% http://www.latex-project.org/lppl.txt
% and version 1.1 or later is part of all distributions of LaTeX
% version 1999/06/01 or later.
%
% DoX consists of the files listed in the file `README'.
%
%
% Commentary:
%
% Contents management by FCM version 0.1.
%
%
% Code:
%
%<*driver>
\documentclass[a4paper]{ltxdoc}
\usepackage{xspace}
\makeatletter
  \def\@currname{dox}
  \def\@currext{inc}
\makeatother
\input{header.inc}
% \OnlyDescription
\CodelineIndex
% \RecordChanges
\begin{document}
\DocInput{dox.dtx}
\end{document}
%</driver>
%
% \fi
%
% \catcode`\�=14
% ^^A\CheckSum{880}
% \GetFileInfo{dox.inc}
%% \CharacterTable
%%  {Upper-case    \A\B\C\D\E\F\G\H\I\J\K\L\M\N\O\P\Q\R\S\T\U\V\W\X\Y\Z
%%   Lower-case    \a\b\c\d\e\f\g\h\i\j\k\l\m\n\o\p\q\r\s\t\u\v\w\x\y\z
%%   Digits        \0\1\2\3\4\5\6\7\8\9
%%   Exclamation   \!     Double quote  \"     Hash (number) \#
%%   Dollar        \$     Percent       \%     Ampersand     \&
%%   Acute accent  \'     Left paren    \(     Right paren   \)
%%   Asterisk      \*     Plus          \+     Comma         \,
%%   Minus         \-     Point         \.     Solidus       \/
%%   Colon         \:     Semicolon     \;     Less than     \<
%%   Equals        \=     Greater than  \>     Question mark \?
%%   Commercial at \@     Left bracket  \[     Backslash     \\
%%   Right bracket \]     Circumflex    \^     Underscore    \_
%%   Grave accent  \`     Left brace    \{     Vertical bar  \|
%%   Right brace   \}     Tilde         \~}
%
% \MakeShortVerb{\|}
%
% \makeatletter
% \def\ps@mystyle{
%   \def\@oddfoot{\hfil\thepage\hfil}
%   \def\@evenfoot{\hfil\thepage\hfil}
%   \def\@evenhead{\hfil\slshape\leftmark}
%   \def\@oddhead{\slshape\rightmark\hfil}}
% \makeatother
%
% \newcommand\dox{\textsf{DoX}\xspace}
% \newcommand\packagecopyright{%
%   Copyright \copyright{} 2009 Didier Verna}
%
% \pagestyle{mystyle}
% \markright{\hskip-.5\marginparwidth \dox \fileversion{} (\filedate)}
%
% \date{\texttt{\fileversion{} (\filedate)}}
% \title{\dox{} -- \textsf{\textbf Doc}, \textbf only e\textbf Xtended}
% \author{Didier Verna\\
%   \texttt{mailto:didier@lrde.epita.fr}\\
%   \texttt{http://www.lrde.epita.fr/\~{}didier/}}
% \maketitle
%
%
% \begin{abstract}
%   The \texttt{doc} package provides \LaTeX{} developers with means to
%   describe the usage and the definition of new macros and environments.
%   However, there is no simple way to extend this functionality to other
%   items (options or counters for instance). The \texttt{dox} package is
%   designed to circumvent this limitation.
% \end{abstract}
%
% ^^A \tableofcontents
%
% \section{Usage}
% \emph{Note: we assume that you know about \texttt{doc}'s \cs{DescribeMacro},
%   \cs{DescribeEnv} and all other associated commands and environments}.
%
% \medskip\noindent
% \DescribeMacro{\doxitem}\marg{funcsname}\marg{singular}\marg{plural}\\
% The \dox style provides a single user-level command to create new items with
% functionalities equivalent to what \texttt{doc} provides for macros and
% environments. Perhaps the simplest way to describe how it works is to give
% an example. Suppose you would like to describe package options. Here is what
% you need to do:
% \begin{verbatim}
% \usepackage{dox}
% \doxitem{Option}{option}{options}
% \end{verbatim}
% \dox then creates the following API for you:
% \begin{itemize}
% \item \cs{DescribeOption}
% \item the \texttt{option} environment
% \item \cs{PrintDescribeOption}
% \item \cs{PrintOptionName}
% \item \cs{SpecialMainOptionIndex}
% \item \cs{SpecialOptionIndex}
% \end{itemize}
%
% That's about it.
% \StopEventually{\par Well, I think that's it. Enjoy using \dox!
%   \vfill\hfill\small \packagecopyright{}.}
%
% \section{Implementation}
% \subsection{Preamble}
%    \begin{macrocode}
%<dox>\NeedsTeXFormat{LaTeX2e}
%<*header>
\ProvidesPackage{dox}[2009/09/11 v1.0 Extensions to the doc package]

%</header>
%    \end{macrocode}
% \subsection{Real job execution}
% \begin{macro}{\doxm@cro@}
%   \marg{item}\marg{contents}\\
%   In \texttt{doc.sty}, the macro and environment environments go through the
%   \cs{m@cro@} command which implements specific parts by testing a boolean
%   condition as its first argument. This mechanism is not extensible, so I
%   have to hack away a more generic version that would work for any new
%   \texttt{dox} item, only which looks pretty much like the original one:
%    \begin{macrocode}
%<*dox>
\long\def\doxm@cro@#1#2{%
  \endgroup%
  \topsep\MacroTopsep\trivlist
  \def\makelabel##1{\llap{##1}}%
  \if@inlabel
    \let\@tempa\@empty
    \count@\macro@cnt
    \loop\ifnum\count@>\z@
      \edef\@tempa{\@tempa\hbox{\strut}}\advance\count@\m@ne%
    \repeat
    \edef\makelabel##1{%
      \llap{\vtop to\baselineskip
	{\@tempa\hbox{##1}\vss}}}%
    \advance\macro@cnt\@ne
  \else
    \macro@cnt\@ne
  \fi
  \edef\@tempa{%
    \noexpand\item[%
%    \end{macrocode}
% That's the first modification:
%    \begin{macrocode}
      \expandafter\noexpand\csname Print#1Name\endcsname{\string#2}]}%
  \@tempa
  \global\advance\c@CodelineNo\@ne
%    \end{macrocode}
% And that's the second one:
%    \begin{macrocode}
  \@nameuse{SpecialMain#1Index}{#2}\nobreak
  \global\advance\c@CodelineNo\m@ne
  \ignorespaces}

%    \end{macrocode}
% \end{macro}
% \subsection{API creation}
% \begin{macro}{\doxitem}
%   \marg{funcsname}\marg{singular}\marg{plural}
%    \begin{macrocode}
\newcommand\doxitem[3]{%
%    \end{macrocode}
% The \cs{Print}\meta{item}\texttt{Name} macro:
%    \begin{macrocode}
  \@ifundefined{Print#1Name}{%
    \expandafter\def\csname Print#1Name\endcsname##1{%
      \strut\MacroFont\string##1\ }}{}
%    \end{macrocode}
% The \cs{SpecialMain}\meta{item}\texttt{Index} macro:
%    \begin{macrocode}
  \expandafter\def\csname SpecialMain#1Index\endcsname##1{%
    \@bsphack%
    \special@index{%
      ##1\actualchar{\string\ttfamily\space##1} (#2)\encapchar main}%
    \special@index{%
      #3:\levelchar##1\actualchar%
      {\string\ttfamily\space##1}\encapchar main}%
    \@esphack}
%    \end{macrocode}
% The \cs{PrintDescribe}\meta{item} macro:
%    \begin{macrocode}
  \@ifundefined{PrintDescribe#1}{%
    \expandafter\def\csname PrintDescribe#1\endcsname##1{%
      \strut\MacroFont##1\ }}{}
%    \end{macrocode}
% The \cs{Special}\meta{item}\texttt{Index} macro:
%    \begin{macrocode}
  \expandafter\def\csname Special#1Index\endcsname##1{\@bsphack
    \index{##1\actualchar{\protect\ttfamily##1}
      (#2)\encapchar usage}%
    \index{#3:\levelchar##1\actualchar{\protect\ttfamily##1}\encapchar
      usage}\@esphack}
%    \end{macrocode}
% The \cs{Describe@}\meta{item} macro:
%    \begin{macrocode}
  \expandafter\def\csname Describe@#1\endcsname##1{%
    \endgroup
    \marginpar{\raggedleft\@nameuse{PrintDescribe#1}{##1}}%
    \@nameuse{Special#1Index}{##1}\@esphack\ignorespaces}
%    \end{macrocode}
% The \cs{Describe}\meta{item} macro:
%    \begin{macrocode}
  \expandafter\def\csname Describe#1\endcsname{%
    \leavevmode\@bsphack\begingroup\MakePrivateLetters
    \@nameuse{Describe@#1}}
%    \end{macrocode}
% The \meta{item} environment:
%    \begin{macrocode}
  \expandafter\def\csname #2\endcsname{%
    \begingroup
      \catcode`\\12
      \MakePrivateLetters\doxm@cro@{#1}}
  \expandafter\let\csname end#2\endcsname\endtrivlist}

%</dox>
%    \end{macrocode}
% \end{macro}
% ^^A \PrintChanges
% ^^A \PrintIndex
% ^^A \Finale
%
% ^^A dox.dtx ends here
